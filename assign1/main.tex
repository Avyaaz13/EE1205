% \iffalse
\let\negmedspace\undefined
\let\negthickspace\undefined
\documentclass[journal,12pt,twocolumn]{IEEEtran}
\usepackage{cite}
\usepackage{amsmath,amssymb,amsfonts,amsthm}
\usepackage{algorithmic}
\usepackage{graphicx}
\usepackage{textcomp}
\usepackage{xcolor}
\usepackage{txfonts}
\usepackage{listings}
\usepackage{enumitem}
\usepackage{mathtools}
\usepackage{gensymb}
\usepackage{comment}
\usepackage[breaklinks=true]{hyperref}
\usepackage{tkz-euclide} 
\usepackage{listings}
\usepackage{gvv}                                        
\def\inputGnumericTable{}                                 
%\usepackage[latin1]{inputenc}                                
\usepackage{color}                                            
\usepackage{array}                                            
\usepackage{longtable}                                       
\usepackage{calc}                                             
\usepackage{multirow}                                         
\usepackage{hhline}                                           
\usepackage{ifthen}                                           
\usepackage{lscape}

\newtheorem{theorem}{Theorem}[section]
\newtheorem{problem}{Problem}
\newtheorem{proposition}{Proposition}[section]
\newtheorem{lemma}{Lemma}[section]
\newtheorem{corollary}[theorem]{Corollary}
\newtheorem{example}{Example}[section]
\newtheorem{definition}[problem]{Definition}
\newcommand{\BEQA}{\begin{eqnarray}}
\newcommand{\EEQA}{\end{eqnarray}}
\newcommand{\define}{\stackrel{\triangle}{=}}
\theoremstyle{remark}
\newtheorem{rem}{Remark}
\begin{document}

\bibliographystyle{IEEEtran}
\vspace{3cm}

\title{NCERT 12.10 5Q}
\author{EE23BTECH11013 - Avyaaz$^{*}$% <-this % stops a space
}
\maketitle
\newpage
\bigskip

\renewcommand{\thefigure}{\theenumi}
\renewcommand{\thetable}{\theenumi}

\large\textbf{\textsl{Question:}}
In Young’s double-slit experiment using monochromatic light of wavelength $\lambda$, the intensity of light at a point on the screen where path difference is $\lambda$, is $K$ units. What is the intensity of light at a
point where path difference is $\lambda$/3?\\
\large\textbf{\textsl{Solution:}}\\
Given,\\
\hspace*{1cm}Path difference = $\lambda$


Let $I_1$ and $I_2$ be the intensity of two coherent waves. The resultant intensity is given by:

$I_{\text{net}} = I_1 + I_2 + 2\sqrt{I_1I_2}\cos{\phi}$

Here, $\phi$ is the phase difference between two light waves.

Intensities are equal for monochromatic light waves.

\hspace{1cm}$I_1 = I_2$

$\therefore$ $I_{\text{net}} = I_1 + I_1 + 2\sqrt{I_1I_1}\cos{\phi}$

\vspace{0.2cm}

       \hspace{0.2cm}      $I_{\text{net}} = 2I_1 + 2I_1\cos{\phi}$

$\because I_{\text{net}} = K$


          K = $ 2I_1 + 2I_1\cos{\phi}$
 
We know that, 

      Phase difference =$\dfrac{2\pi}{\lambda}$ x Path difference

      \vspace{0.2cm}

$\because$ path difference = $\lambda$

\vspace{0.2cm}

\hspace{1cm}$\phi = \dfrac{2\pi}{\lambda}$  x  $\lambda$

\vspace{0.2cm}

Phase difference = $\phi = 2\pi$

$\therefore K = 2I_1 +2I_1\cos{2\pi}$

\vspace{0.2cm}

        \hspace{0.2cm}    K = 4$I_1$

\vspace{0.2cm}

      \hspace{1cm}  $\therefore    I_1=\dfrac{K}{4}$

\vspace{1cm}

When path difference = $\dfrac{\lambda}{3}$

\vspace{0.3cm}

Phase difference = $ \phi = \dfrac{2\pi}{3} $

Hence,

\hspace*{0.5cm}Resultant intensity,
\[I_{\text{R}} = 2I_1 + 2\sqrt{I_1I_1}\cos{\dfrac{2\pi}{3}}\]

\[I_{\text{R}}= 2I_1 + 2I_1\left(\dfrac{-1}{2}\right)\]

\[I_{\text{R}} = I_1\]

From the above result,
\hspace*{0.5cm}
\begin{align*}
I_1 = \dfrac{K}{4}\\
\therefore I_{\text{R}} = \dfrac{K}{4}
\end{align*}

Hence, the Intensity of light at a point where path difference is $\dfrac{\lambda}{3}$ is $\dfrac{K}{4}$ units.

\bibliographystyle{IEEEtran}
\end{document}
